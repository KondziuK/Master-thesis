\chapter{Rekonstrukcja 3D}
\label{cha:Rekonstrukcja}
W niniejszym rozdziale przedstawiono temat zagadnienia rekonstrukcji 3D. Przedstawiono główne założenia klasycznej stereowizji, kroki konieczne do jej poprawnego przeprowadzenia oraz miejsce kamer zdarzeniowych w nich. Dalej opisano implementowane w kolejnym rozdziale algorytmy. W ostatnim podrozdziale ukazano opis możliwych zastosowań sieci neuronowych w wykonywanej rekonstrukcji 3D.

\section{Stereowizja}


\subsection{Opis sterowizji}
Stereowizja, cytując za słownikiem języka polskiego, to "technika obrazowa umożliwiająca rekonstrukcję scen trójwymiarowych na podstawie obrazów pozyskanych z co najmniej dwóch sensorów optyczno-elektronicznych." W roli wspomnianych sensorów występują najczęściej kamery, a w przypadku niniejszej pracy, kamery zdarzeniowe.

\subsection{Algorytmy stereowizyjne}

\section{Sieci neuronowe}
    \subsection{Opis sieci}
    
    \subsection{implementowane sieci}
Opis sieci

