\chapter{Wprowadzenie}
\label{cha:wprowadzenie}
W rozdziale tym omówiono wstęp do pracy
\section{Cele pracy}
\label{sec:celePracy}
Celem niniejszej pracy było stworzenie systemu umożliwiającego rekonstrukcję informacji 3D na podstawie danych zarejestrowanych przy pomocy kamery bądź też kamer zdarzeniowych. Wykorzystane metody opierać się miały o  standardowe algorytmy stosowane w systemach wizyjnych dostosowanych jednakże do danych z kamer zdarzeniowych, bądź też w oparciu o głębokie albo impulsowe sieci neuronowe. Implementacja algorytmów miała nastąpić przy użyciu języków wysokiego poziomu, takich jak C++, MATLAB, Python.\\
\indent W pierwszym etapie pracy należało dokonać przeglądu literatury naukowej związanej z tematem pracy. Konieczne było zrozumienie różnic pomiędzy standardowymi czujnika wizyjnymi a wykorzystywanymi zdarzeniowymi. Część teoretyczna pracy powinna zawierać porównanie ich pod względem sposobu działania, typowych zastosowań oraz technik przetwarzania danych. Należało umieścić informacje odnośnie historii ich rozwoju, stosowanych technik zapisu danych, 

\section{Zawartość pracy}
\label{sec:zawartoscPracy}


