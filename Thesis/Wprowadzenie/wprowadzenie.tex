\chapter{Wprowadzenie}
\label{cha:wprowadzenie}

Od wielu lat możemy zaobserwować znaczny rozwój techniki w dziedzinie automatyzacji działań człowieka. Roboty, drony oraz inne urządzenia działające autonomicznie bez udziału bądź z możliwie najmniejszym udziałem człowieka jeszcze kilkadziesiąt lat temu jawiły się jako coś niemożliwego, będącego nieosiągalną pieśnią przyszłości. W czasach współczesnych jednakże zajmują one już pokaźną niszę w przeróżnych dziedzinach nauki oraz przemysłu a ich rozwój na chwilę obecną nie jest w żaden sposób zagrożony. Jednakże aby możliwe było poprawne wykonywanie przez wspomniane urządzenia powierzonych im zadań konieczne jest aby mogły one w sposób świadomy orientować się w otaczającej je przestrzeni - innymi słowy, konieczne jest aby używały skutecznych systemów wizyjnych. To właśnie ich zadaniem jest, przy użyciu układów współpracujących ze sobą układów elektronicznych, dokonywać w sposób automatyczny analizy wizyjnej otoczenia na podobieństwo ludzkiego zmysłu wzroku. Pozwala to chociażby właśnie na orientowanie się obiektu w przestrzeni dzięki odtwarzaniu jej przy pomocy narzędzia akwizycji danych - kamery oraz odpowiedniego algorytmu przetwarzającego - tak powstaje rozpatrywana w ramach niniejszej pracy ''Rekonstrukcja informacji 3D". Bardzo istotnym elementem jest właśnie używane narzędzie akwizycji - w rozpatrywanej pracy używam nowego typu kamery, czyli kamery zdarzeniowej która posiada widoczne zalety nad rozwiązaniami dotychczas powszechnie stosowanymi.



\section{Cele pracy}
\label{sec:celePracy}

Celem niniejszej pracy było stworzenie systemu umożliwiającego rekonstrukcję informacji 3D na podstawie danych zarejestrowanych przy pomocy kamery bądź też kamer zdarzeniowych. Wykorzystane metody opierać się miały o  standardowe algorytmy stosowane w systemach wizyjnych dostosowanych jednakże do danych z kamer zdarzeniowych, bądź też w oparciu o głębokie albo impulsowe sieci neuronowe. Implementacja algorytmów miała nastąpić przy użyciu języków wysokiego poziomu, takich jak C++, MATLAB, Python.\\
\indent W pierwszym etapie pracy należało dokonać przeglądu literatury naukowej związanej z tematem pracy. Konieczne było zrozumienie różnic pomiędzy standardowymi czujnika wizyjnymi a wykorzystywanymi zdarzeniowymi. Część teoretyczna pracy powinna zawierać porównanie ich pod względem sposobu działania, typowych zastosowań oraz technik przetwarzania danych. Należało umieścić informacje odnośnie historii ich rozwoju, stosowanych technik zapisu danych jak i  popularnych dostępnych na rynku konstrukcji. Opisać należało także biblioteki pozwalające przekształcić sekwencje obrazów zarejestrowanych klasyczną kamerą na ciąg zdarzeń oraz charakterystykę badanych sieci neuronowych i zastosowanych algorytmów.\\
\indent Po dokonaniu ww.działań, na podstawie przeprowadzonej analizy literatury, należało wybrać odpowiednie algorytmy bądź też architektury sieci neuronowych przy założeniu, że podstawowym stosowanym wariantem będzie ten z użyciem dwóch kamer zdarzeniowych. Wybrane rozwiązania należało zaimplementować, nauczyć i poddać ewaluacji na dostępnych zbiorach testowych.\\
\indent W ostatnim etapie należało podjąć próbę optymalizacji czasowej oraz jakościowej procesu rekonstrukcji 3D. Uzyskane wyniki przedyskutować oraz wskazać możliwe dalsze kroki rozwoju.

\section{Zawartość pracy}
\label{sec:zawartoscPracy}

Niniejsza praca została podzielona na X rozdziałów. Pierwszy z nich opisuje osadzenie pracy w kontekście rozwoju techniki systemów wizyjnych. Ponadto opisano w nim zarówno cele jak i zawartość pracy. W rozdziale drugim przedstawiono czujnik będący przedmiotem rozważań pracy - kamerę zdarzeniową. Przedstawiono charakterystykę sprzętu, porównano do kamer klasycznych oraz przeanalizowano dostępność komercyjną. Opisano również stosowane biblioteki ułatwiające pracę z nimi jak i używane zestawy danych. Rozdział trzeci opisuje rekonstrukcję 3D - ukazuje teorię stojącą za stereowizją jak również klasyczne algorytmy rekonstrukcji oraz sieci neuronowe. W kolejnym, 4 rozdziale, opisano implementacje poszczególnych algorytmów.Rozdział piąty służy szczegółowemu przedyskutowaniu otrzymanych wyników. W ostatnim rozdziale zawarto podsumowanie niniejszej pracy jak i wskazano dalsze możliwe kroki w jej rozwoju.


